\documentclass[11pt,a4paper]{article}
\usepackage[utf8]{inputenc}
\usepackage[french]{babel}
\usepackage[T1]{fontenc}
\usepackage{amsmath}
\usepackage{amsfonts}
\usepackage{amssymb}
\usepackage{graphicx}
\usepackage[left=2cm,right=2cm,top=2cm,bottom=2cm]{geometry}
\author{Juckler Christian 3TL1}
\title{}
\begin{document}
\begin{center}
{\bfseries Résumé opérationnel}\\
{\bfseries Installation, configuration et comparaison de solutions professionnelles d'accès à distance}\\
Auteur : Juckler Christian 3TL1\\
Rapporteur : Virginie Van den Schrieck
\end{center}
Les accès distants sont un enjeu essentiel des entreprises à l'heure actuelle.
Mais la mise à disposition des ressources internes vers l'extérieur n'est pas sans risques.
Il est donc important de sécuriser les connexions entre les clients distants et les ressources internes.
Cet objectif doit se faire en ne pénalisant pas l'utilisateur.\\

Ce travail a été effectué en réalisant une infrastructure similaire à celle du PME et en utilisant les technologies d'accès distant proposées par Juniper, Cisco et Microsoft.\\

La première étape consiste à configurer un domaine répondant aux besoins standards d'une entreprise, à savoir un serveur mail, un serveur de fichier et un intranet.
L'infrastructure est composée d'un "domain controller" dont le domaine est \texttt{labotfe.be}, d'un serveur exchange, d'un intranet et d'un "DFS Namespace". 

Le deuxième étape est la configuration des passerelles et l'intégration de la passerelle dans le réseau de l'entreprise. 
La solution Juniper utilise une passerelle SLL placée dans une DMZ.
La solution Windows utilise un serveur dédié pour le "Remote Access", qui est aussi placé dans une DMZ.
La solution Cisco intègre la passerelle dans le firewall.
Les configurations sont réalisées à l'aide des guides d'administrations fournis par les fabricants.

La troisième étape est la réalisation des tests pour mener à une comparaison des différentes solutions.

Lors de ce travail, j'ai réalisé un comparatif des solutions Cisco, Juniper et Microsoft d'accès distants.
Pour les solutions Cisco et Juniper, j'ai séparé les résultats selon la méthode d'accès.
Pour ces deux solutions, le client peut se connecter soit via un portail Web, soit via un client logiciel.
Quant à la solution Microsoft, elle n'utilise pas de client logiciel, ni de portail.\\

Le comparatif réalisé montre que les solutions sont similaires en termes de fonctionnalités.
En effet, toutes les solutions testées arrivent à accéder à l'ensemble des ressources prévues. 
Les différences se situent au niveau de l'interface graphique et au niveau de la facilité de configuration.

La solution Juniper est la plus agréable à utiliser. 
Son interface graphique est intuitive, et sa configuration est claire.

La solution Microsoft tire son épingle du jeu et est une bonne surprise.
Elle est relativement simple à configurer, on peut lui reprocher un manque de transparence sur son fonctionnement.
En effet, toutes les opérations sont effectuées par le système sans que l'utilisateur intervienne.
On peut aussi pointer qu'elle n'est utilisable qu'avec des ordinateurs utilisant des versions spécifiques des systèmes d'exploitation Windows 7, 8 et 8.1.
Pour accéder aux mails avec un smartphone, il faut donc utiliser une autre technologie.

La solution Cisco déçoit car son interface est mal conçue et le logiciel est buggué.
La configuration est lourde, et les modifications doivent être répétées à plusieurs endroits pour qu'elles soient totalement effectives.\\

L'un des avantages de la solution Microsoft par rapport aux autres solutions réside dans le fait que les machines bien que distantes sont toujours liées au réseau de l'entreprise.
Le gestionnaire du parc informatique est donc capable de patcher les machines distantes de la même manière que si elles étaient en interne.
Cette capacité réduit le risque que ces machines soient out-of-date sur les mises à jour de sécurité critiques.
\end{document}